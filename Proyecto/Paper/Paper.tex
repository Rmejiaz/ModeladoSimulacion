\documentclass[conference]{IEEEtran}

\usepackage[spanish]{babel}
\usepackage[utf8]{inputenc}
\usepackage{amsmath}
\usepackage{graphicx}
\usepackage[colorinlistoftodos]{todonotes}
\usepackage{tikz}
\usepackage{url}
\usepackage{multicol}
\usepackage{csquotes}
\usepackage{mathtools}
\usepackage{float}
\usetikzlibrary{shapes, shadows, arrows}
\pagenumbering{arabic}
\usepackage{lastpage}
\usepackage{fancyhdr}
\usepackage{hyperref}
\pagestyle{fancy}
\fancyhf{}
\rfoot{Página \thepage \hspace{1pt} de \pageref{LastPage}}


\DeclareMathOperator{\diag}{diag}

\providecommand{\keywords}[1]
{
  \small	
  \textbf{\textit{Palabras Clave---}} #1
}


\begin{document}
    

\title{Modelo SEIR mejorado del COVID-19 y sus dinámicas}

\author{Rafael Mejía Zuluaga \\\\Departamento de Ingeniería Eléctrica, Electrónica y Computación \\ Universidad Nacional de Colombia sede Manizales \\ rmejiaz@unal.edu.co}

\maketitle


\begin{abstract}
    El presente documento es un análisis del artículo \textit{SEIR modeling of the
    COVID-19 and its dynamics} [1]. Primero, se hace una breve descripción del modelo SEIR 
    clásico y del modelo presentado por los autores, el cual es una versión mejorada del mismo.
    Luego, se presentan los resultados y el análisis de algunas simulaciones realizadas en 
    \textit{Python} y por último se exponen algunas conclusiones.  
    \newline
\end{abstract}

\keywords{Modelo SEIR, Modelado Epidemiológico, COVID-19, Bifurcaciones, Predicciones, Caos}

\section{Introducción}
Los modelos SEIR son ampliamente utilizados en el campo de la epidemiología para 
modelar el compartamiento de enfermedades infecciosas como lo es en la actualidad 
el COVID-19. Utilizando estos modelos es posible hacer predicciones con respecto 
a la evolución y propagación de enfermedades infecciosas dentro de una población, que 
luego pueden ser utilizadas por los entres gubernamentales en la toma de desiciones.
\\\\
El modelo SEIR clásico parte de un principio fundamental el cual consiste en dividir a
la población total en cuatro grupos diferentes: $S$ (susceptibles), $E$ (expuestos),
$I$ (infectados) y $R$ (recuperados). La idea es que a media que pasa el tiempo, 
todos los individuos de la población van a pertenecer a todos los grupos, siguiendo la 
ruta $S \rightarrow E \rightarrow I \rightarrow R$. Las variables del sistema son
prescisamente la cantidad de personas en cada uno de estos grupos. El modelo también 
parte de la base que la cantidad total de individuos $N$ se mantiene constante (no toma en
cuenta los nacimientos ni las muertes), por lo que la cantidad de personas que salen de
un grupo necesariamente deben entrar a otro de los grupos, y en todo momento se cumple
$N = S + E + I + R$.
\section{Modelo SEIR mejorado}

El modelo propuesto por He et al. [1] es una versión ampliada del modelo SEIR clásico, en el cual 
se incluyen dos variables más: $H$ (hospitalizados) y $Q$ (en cuarentena). Además, se
divide la categoría de infectados en dos grupos: $I_1$ (infecados sin intervención)
e $I_2$ (infectados con intervención).
\\\\
A diferencia del modelo SEIR clásico, en este modelo se tienen dos canales principales,
el primero es $S \rightarrow E \rightarrow I_1 \rightarrow R$ y el segundo
$S \rightarrow Q \rightarrow I_2 \rightarrow H \rightarrow R$. El primer 
caso ilustra el comportamiento natural de una pandemia y equivale al SEIR clásico, 
mientras que el segundo hace referencia a los mecanismos de control impuestos por los
gobiernos tales como cuarentenas y hospitazaciones. Por último, otra diferencia 
importante de este modelo con respecto al SEIR clásico es que en este los individuos
pueden pasar de $R$ nuevamente a $S$, pues se ha demostrado que es posible
contagiarse más de una vez. A continuación se muestra el modelo propuesto:

\begin{equation}   
    \begin{aligned}
    \left\{
        \begin{array}{l} 
        \dot{S} = - \frac{S}{N}\left( {{\beta _1}{I_1} + {\beta _2}{I_2} + \chi E} \right) + {\rho _1}Q - {\rho _2}S + \alpha R
        \\ 
        \dot{ E} = \frac{S}{N}\left( {{\beta _1}{I_1} + {\beta _2}{I_2} + \chi E} \right) - {\theta _1}E - {\theta _2}E
        \\ 
        {\dot{I}}_1 = {\theta _1}E - {\gamma _1}{I_1}
        \\ 
        \dot{I}_2 = {\theta _2}E - {\gamma _2}{I_2} - \varphi {I_2} + \lambda \left( \varLambda + Q \right) 
        \\ 
        \dot{R} = {\gamma _1}{I_1} + {\gamma _2}{I_2} + \phi H - \alpha R
        \\ 
        \dot{H} = \varphi {I_2} - \phi H
        \\ 
        \dot{Q} = \varLambda + {\rho _2}S - \lambda \left( {\varLambda + Q} \right) - {\rho _1}Q 
    \end{array} 
    \right.
    \end{aligned}
\end{equation}

En los cuadros \ref{var_desc} y \ref{sys_pars} se pueden ver las descripciones de las
variables y los parámetros del modelo respectivamente.

\begin{table}[h]
    \centering
    \begin{tabular}{ll}
    \hline
    Variable & Descripción                 \\ \hline
    $S$      & Susceptibles                \\ 
    $E$      & Expuestos                   \\ 
    $I_1$    & Infectados sin intervención \\ 
    $I_2$    & Infectados con intervención \\ 
    $R$      & Recuperados                 \\ 
    $Q$      & En cuarentena               \\ 
    $H$      & Hospitalizados              \\ \hline
    \end{tabular}
    \caption{Descripción de las variables del sistema}
    \label{var_desc}
\end{table}


\begin{table}[h]
    \begin{tabular}{ll}
    \hline
    Parámetros            & Descripción                                                        \\ \hline
    $\alpha$              & Tasa de inmunidad temporal                                         \\ 
    $\beta_1, \beta_2$    & Tasa de transmisión por contacto con la clase de infecados         \\ 
    $\chi$                & Probabilidad de transmisión por contacto con individuos expuestos  \\ 
    $\theta_1 , \theta_2$ & Tasa de transición de individuos a la clase de infectados          \\ 
    $\gamma_1 , \gamma_2$ & Tasa de recuperación de infectados sintomáticos a recuperados      \\ 
    $\varphi$                & Tasa de transición de infectados con síntomas a hospitalizados     \\ 
    $\phi$                & Tasa de recuperación de individuos infectados en cuarentena        \\ 
    $\lambda$             & Tasa de transición de individuos en cuarentena a infectados        \\ 
    $\rho_1 , \rho_2$     & Tasa de transición entre susceptibles y en cuarentena y vice versa \\ 
    $\Lambda$             & Entrada externa de otros países o regiones                         \\ \hline
    \end{tabular}
    \caption{Descripción de los parámetros del sistema}
    \label{sys_pars}
\end{table}



La figura \ref{block_diagram} muestra un diagrama de flujo del modelo con los diferentes canales.

\begin{figure}[H]
    \centering
    \includegraphics[width=8.5cm]{../Figures/Model_flowchart.pdf}
    \caption{Diagrama de flujo del modelo}
    \label{block_diagram}
\end{figure}


Como puede verse, el modelo tiene una gran cantidad de parámetros, los cuales deben
ser cuidadosamente seleccionados según las dinámicas de cada región. Por ejemplo, los 
mecanismos de control impuestos por los gobiernos pueden variar en cada país.
\\\\
En el caso del artículo original, se utiliza el algoritmo PSO (\textit{particle swarm optimization})
para estimar los parámetros del modelo de acuerdo a un conjunto de datos de la provincia
de Hubei, en China, de donde también se toman las condiciones iniciales del sistema.
\\\\
Los cuadros \ref{ic1} y \ref{p1} muestran las condiciones iniciales y los parámetros
utilizados.
\\\\
\begin{table}[h]
    \centering
    \begin{tabular}{ll}
    \hline
    Parámetros  & Valores                 \\ \hline
    $\alpha$    & $1.2048 \times 10^{-4}$ \\ 
    $\beta_1$   & $1.0538 \times 10^{-1}$ \\ 
    $\beta_2$   & $1.0538 \times 10^{-1}$ \\ 
    $\chi$      & $1.6221 \times 10^{-1}$ \\ 
    $\theta_1 $ & $9.5 \times 10^{-4}$    \\ 
    $\theta_2$  & $3.5412 \times 10^{-2}$ \\ 
    $\gamma_1$  & $8.5 \times 10^{-3}$    \\ 
    $\gamma_2$  & $1.0037 \times 10^{-3}$ \\ 
    $\lambda$   & $9.4522 \times 10^{-2}$ \\ 
    $\rho_1 $   & $2.8133 \times 10^{-3}$ \\ 
    $\rho_2$    & $1.2668 \times 10^{-1}$ \\ \hline
    \end{tabular}
    \caption{Parámetros utilizados estimados a partir del algoritmo PSO}
    \label{p1}
\end{table}

\begin{table}[h]
    \centering
    \begin{tabular}{ll}
    \hline
    Variable  & Valor                  \\ \hline
    $N$       & $6.5563 \times 10 ^ 4$ \\ 
    $E$       & $5077$                 \\ 
    $I_1$     & $I_2 \times 0.01$      \\ 
    $I_2$     & $729$                  \\ 
    $H$       & $658$                  \\ 
    $R$       & $32$                   \\ 
    $Q$       & $4711$                 \\ 
    $\Lambda$ & $10$                   \\ \hline
    \end{tabular}
    \caption{Condiciones iniciales del sistema}
    \label{ic1}
\end{table}
La figura \ref{pvi_1} muestra una simulación del problema de valor inicial, utilizando
los mismos parámetros y valores iniciales del artículo. 

\begin{figure}[h]
    \centering
    \includegraphics[width=10cm]{../Figures/ivp_1.png}
    \caption{Solución numérica del problema de valores iniciales realizada en Python}
    \label{pvi_1}
\end{figure}

\section{Infección estocástica y estacional}

La estacionalidad es ampliamente utilizada en modelos epidemiológicos. A pesar de que no se sabe
a ciencia cierta su efecto en la propagación del COVID-19 (por lo menos a la fecha
de publicación del artículo original), se intenta introducir estacionalidad al sistema para analizar el caos, 
por fines netamente académicos.
\\\\
Además de esto, existen muchos factores imprededecibles que afectan las diferentes tazas de contagio,
por lo que es conveniente estudiar también el efecto del ruido en estas.
\\\\
En particular, se analizan 3 casos diferentes, introduciendo estacionalidad y ruido a diferentes parámetros.
\\\\
\subsubsection{Caso 1}

El parámetro $\beta_1$ contiene estacionalidad e infección estocástica, y las tres tazas de contacto e infección se definen como:
\\\\
$\beta_2 = 30.03, \ \chi = 30.40$
\\
$\beta_1(t) = \beta_0(1 + \varepsilon_1 \sin(2\pi t) + \varepsilon_2 \xi(t)$
\\\\
Donde $\beta_0 = 2 \times \beta_1 = 60$, $\varepsilon_1$ y $\varepsilon_2$ son grados de infección 
estacional y estocástica respectivamente. $\langle \xi(t) \rangle$ es ruido blanco gausiano, 
el cual tiene las propiedadees de $\langle \xi(t) \rangle = 0$ y $\langle \xi(t), \xi(\tau) \rangle = \delta(t - \tau)$
\\\\
La figura \ref{case_1} muestra algunas simulaciones de este caso variando $\varepsilon_1, \varepsilon_2$
y $\alpha$. En particular se analizan 3 casos: uno sin infección estacional ni estocástica, otro con infección
unicamente estacional, y por ultimo uno con infección estacional y estocástica. 
\\
Para el primer caso, puede verse como el sistema se estabiliza en una región confinada, mientras 
que con infección estacional y estocástica el sistema presenta oscilaciones alrededor de un punto. En ambos
casos las órbitas no son únicas; por el contrario existen diferentes órbitas (caos) las cuales en el caso 
de infección estacional no se hacen tan evidentes por su proximidad, mientras que en el caso estacional y
estocástico se pueden apreciar claramente numerosas órbitas.
\\\\
\begin{figure}[h]
    \centering
    \includegraphics[width=8cm]{../Figures/case_1.png}
    \caption{Diagramas de fase y en el tiempo para el caso 1. La primera fila representa el caso en el que $\varepsilon_1 = 0$
    y $\varepsilon_2 = 0$ (sin infección estacional ni estocástica). La segunda fila para $\varepsilon_1 = 0.8$ y $\varepsilon_2 = 0$ (infeccón 
    estacional pero no estocástica) y la tercera para $\varepsilon_1 = 0.8$ y $\varepsilon_2 = 0.2$ 
    (infección estacional y estocástica)}
    \label{case_1}
\end{figure}

\subsubsection{Caso 2}

El parámetro $\beta_2$ contiene infección estocástica y estacional, mientras que las otras dos tazas de infección permanecen constantes. Los parámetros quedan definidos como:
\\\\
$\beta_1 = 30, \ \chi = 30.40$
\\
$\beta_2(t) = \beta_0(1+\varepsilon_1 \sin(2 \pi t) + \varepsilon_2 \xi(t))$
\\\\
Donde $\beta_0 = 2 \times \beta_2 = 60$, $\varepsilon_1$ y $\varepsilon_2$ son grados de infección estocástica y estacional respectivamente.
\\\\
Al igual que en el caso 1, en este se varían los parámetros de diferentes formas para analizar diferentes
escenarios, como se puede ver en la figura \ref{case_2}. Dado que ya se demostó en el caso 1 que el sistema sin infección estacional ni estocástica no 
presenta caos, este caso se obvia. Para los otros dos casos (solo estacional y tanto estacional como estocástica)
se utilizan diferentes valores de $\alpha$. En general, los resultados son muy similares a los del caso 1. 
En ambos casos se evidencia presencia de caos, aunque claramente y por obvias razones más marcada en el
caso de infección estocástica. A pesar de que las figuras lucen todas iguales, hay variaciones en las escalas
por lo que se puede concluir que el parámetro $\alpha$ afecta la amplitud de las oscilaciones generadas por 
la estacionalidad de la infección.

\begin{figure*}[h]
    \centering
    \includegraphics[width=18cm]{../Figures/case_2.png}
    \caption{Diagramas de fase para diferentes valores de $\varepsilon_1, \varepsilon_2$ y $\alpha$. 
    La primera fila con $\varepsilon_1 = 0.8$ y $\varepsilon_2 = 0$ (infección estacional
    pero no estocástica), mientras que en la segunda fila se encuentran $\varepsilon_1 = 0.8$ y $\varepsilon_2 = 0.2$
    (infección estacional y estocástica). En cuanto al parámetro $\alpha$, este varía a lo largo de las
    columnas en el siguiente orden: $\alpha = [0.02 ; 0.03 ; 0.04747 ; 0.08]$}
    \label{case_2}
\end{figure*}



\subsubsection{Caso 3}

El parámetro $\chi$ contiene infección estocástica y estacional, y $\beta_1$ y $\beta_2$ son constanes.
En este caso, los parámetros estarían dados por:
\\\\
$\beta_1 = 30, \ \beta_2(t) = 30$
\\
$\chi = \chi_0(1 + \varepsilon_1 \sin(2 \pi t) + \varepsilon_2 \xi(t))$
\\\\
Donde $\chi_0 = 2\chi = 60.8$, $\varepsilon_1$ y $\varepsilon_2$ son grados de infección estocástica y estacional respectivamente.
\\\\
Como se puede ver en la figura \ref{case_3}, se realizan simulaciones para 3 casos diferentes variando $\alpha, 
\varepsilon_1$ y $\varepsilon_2$. Al igual que con los dos casos anteriores, en este también se presentan
diferentes tipos de órbitas, siendo de mayor amplitud la de mayor $\alpha$. Esto se puede ver tanto en el
diagrama de fase como en la solución en el tiempo. Además, también es evidente que el caso en el que $\alpha = 0.08$
el sistema se estabiliza mucho más rápido que los otros dos.
\begin{figure*}[h]
    \centering
    \includegraphics[width=18cm]{../Figures/case_3.png}
    \caption{Diagramas de fase y en el tiempo para: $\varepsilon_1 = 0.8, \varepsilon_2 = 0$
    y $\alpha = 0.01333$ (infección estacional) ; $\varepsilon_1 = 0.8, \varepsilon_2 = 0.2$
    y $\alpha = 0.0133$ (infección estacional y estocástica) ; $\varepsilon_1 = 0.8, \varepsilon_2 = 0.2$
    y $\alpha = 0.08$ (infección estacional y estocástica)}
    \label{case_3}
\end{figure*}
\\\\
En todos los casos, el ruido blacno gausiano se simuló utilizando una variable aleatoria
con función de distribución normal, con $\mu = 0$ y $\sigma = 0.1$.
\\\\
La tabla \ref{chaos_params} contiene los valores de los demás parámetros utilizados en esta sección. Los cuales pueden
parecer demasiado grandes para tratarse de un modelo epidemiológico, sin embargo estos pueden tener sentido, 
ya que dentro del modelo se multiplica por $S/N$. Estos parámetros no son los mismos que se estimaron con el 
algortimo PSO, sino que se seleccionaron porque se encontró que con estos puede encontrarse caos en el sistema.
\\\\
Por último cabe resaltar que para los diagramas de fase se graficaron únicamente los últimos valores de la simulación,
para eliminar así el transotorio de esta. De lo contrario, por temas de escala no sería posible apreciar las órbitas
que aparecen en el estado estacionario.
\begin{table}[h]
    \centering
    \begin{tabular}{ll}
    \hline
    Parámetros  & Valores    \\ \hline
    $\theta_1 $ & $20.054$    \\ 
    $\theta_2$  & $20.12$ \\ 
    $\gamma_1$  & $26$    \\ 
    $\gamma_2$  & $26$ \\ 
    $\varphi$   & $0.00009$ \\
    $\phi$      & $0.8$  \\
    $\lambda$   & $0.4$ \\
    $\rho_1 $   & $1/14$ \\ 
    $\rho_2$    & $0.002$ \\
    $\Lambda$    & $10$  \\ \hline
    \end{tabular}
    \caption{Parámetros utilizados para simular los diferentes casos de infección 
    estacional y estocástica}
    \label{chaos_params}
\end{table}




\section{Conclusiones}

\begin{itemize}
\item Los modelos SEIR pueden ser de gran utilidad para predecir el avance de una epidemia en una región determinada.
\item La estimación de los parámetros de acuerdo a cada región es de vital importancia para obtener resultados útiles e interpretables.
\item El modelo en cuestión presenta caos pero únicamente para los casos en los cuales se añaden tazas de infección
      estacionales y estocásticas, y las oscilaciones que se tienen son en realidad forzadas.

\end{itemize}
\section*{Anexo 1: Cuaderno con Simulaciones}

Todas las simulaciones y códigos pueden ser consultados en el siguiente enlace: \href{https://github.com/Rmejiaz/ModeladoSimulacion/blob/main/Cuadernos/Proyecto.ipynb}{https://github.com/Rmejiaz/ModeladoSimulacion/blob/main/Cuadernos/Proyecto
.ipynb}


\section{Referencias}

\begin{enumerate}
    \item He, S., Peng, Y. \& Sun, K. SEIR modeling of the COVID-19 and its dynamics. Nonlinear Dyn 101, 1667–1680 (2020). \href{https://doi.org/10.1007/s11071-020-05743-y}{https://doi.org/10.1007/s11071-020-05743-y}
\end{enumerate}

\end{document}
